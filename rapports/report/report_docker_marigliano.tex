% inspired from: https://github.com/SnipyJulmy/hesso-latextemplate-lab
\documentclass[11pt,a4paper,oneside]{report}
\usepackage[margin=2cm]{geometry}
\usepackage[utf8]{inputenc}
\usepackage[T1]{fontenc}
\usepackage[french]{babel}
\usepackage{minted}
\usepackage{titlesec}
\usepackage[pdftex]{graphicx} % graphics importing
\usepackage{titling} % can use \theauthor \thetitle
\usepackage{parskip} % remove first line indenting in a section
\usepackage{microtype} % typographic improvements
\usepackage[defaultlines=3,all]{nowidow}
\usepackage[toc,page]{appendix}
\usepackage{verbatim}
\usepackage{float}

% Header and Footer
\usepackage{fancyhdr}

% style for all normal pages
\fancypagestyle{normal}{
\fancyhf{}
\setlength\headheight{14pt}
\lhead[]{Docker and Embedded systems}
\chead[]{}
\rhead{\includegraphics[width=3cm]{img/mse_logo}}
\cfoot[]{\thepage}
\renewcommand{\headrulewidth}{0.4pt}% Default \headrulewidth is 0.4pt
\renewcommand{\footrulewidth}{0.4pt}% default is 0pt
}

% style for history
\fancypagestyle{historystyle}{
\setlength\headheight{14pt}
\lhead[]{Docker and Embedded systems}
\chead[]{}
\rhead{\includegraphics[width=3cm]{img/mse_logo}}
\renewcommand{\headrulewidth}{0.4pt}% Default \headrulewidth is 0.4pt
\renewcommand{\footrulewidth}{0pt}
}

\usepackage[hyphens]{url} % line wrap urls
\usepackage{hyperref}

% Version history
\usepackage{vhistory}

\newminted{bash}{xleftmargin=20pt, linenos=true, breaklines=true, frame=single, framesep=6pt, tabsize=2, fontfamily=courier, fontsize=\small}

% Chapter titles
% Remove space before title
\titlespacing{\chapter}{0pt}{*-4}{*3}
% Remove "Chapter N" and use a sans-serif font
\titleformat{\chapter}{\normalfont\huge}{\thechapter.}{20pt}{\huge}
% Change chapter page style
\patchcmd{\chapter}{plain}{fancy}{}{}

\newcommand{\code}[1]{\texttt{#1}} % inline code

\title{Projet de semestre Docker and embedded systems}
\author{Gary \bsc{Marigliano}}

\begin{document}

\begin{titlepage}
\begin{center}

\includegraphics[width=0.6\textwidth]{img/docker_logo}\\[1cm]

\begin{figure}[htbp]
\begin{minipage}[c]{.45\linewidth}
\begin{flushleft}
\includegraphics[width=7cm]{img/mse_logo}
\end{flushleft}
\end{minipage}
\hfill
\begin{minipage}[c]{.45\linewidth}
\begin{flushright}
\includegraphics[height=2cm]{img/logo_hes-so}
\end{flushright}
\end{minipage}
\end{figure}

% Title
\rule{\linewidth}{0.5mm} \\[0.4cm]
{ \huge \bfseries \thetitle \\[0.4cm] }
\rule{\linewidth}{0.5mm} \\[1.5cm]

% Author and supervisor
\noindent
\begin{minipage}{0.4\textwidth}
  \begin{flushleft} \large
    \emph{Auteur :}\\
    \theauthor
  \end{flushleft}
\end{minipage}%
\begin{minipage}{0.4\textwidth}
  \begin{flushright} \large
    \emph{Encadrant :} \\
    Jean-Roland \bsc{Schüler}
  \end{flushright}
\end{minipage}

\vfill

\noindent
\begin{minipage}{0.4\textwidth}
  \begin{flushleft} \large
    \emph{Contact :}\\
    gary.marigliano@master.hes-so.ch
  \end{flushleft}
\end{minipage}%
\begin{minipage}{0.4\textwidth}
  \begin{flushright} \large
    \emph{Mandant :} \\
    Haute école d'ingénierie et d'architecture de Fribourg
  \end{flushright}
\end{minipage}

\vfill

% Bottom of the page
{\large Version 0.0.1 \\ \today}

\end{center}
\end{titlepage}

\pagestyle{historystyle}
\begin{versionhistory}  
  \vhEntry{0.0.1}{02.05.16}{\theauthor}{Création du document}
\end{versionhistory}


\pagenumbering{gobble}
\tableofcontents
\pagenumbering{arabic}

\pagestyle{normal}

\chapter{Introduction}

\section{Contexte}\label{contexte}

Ce document s'inscrit dans le cadre du projet de semestre Docker and embedded systems actuellement réalisé par moi-même. Un des buts de ce projet est de cross compiler Docker à partir de ses sources pour produire un binaire exécutable sur un Odroid XU3 (ARMv7). De plus, une partie concernant la sécurité de Docker sera également traitée.

Lien: \url{https://github.com/krypty/docker_and_embedded_systems}

Il est important de noter que la vitesse de développement de Docker est assez hallucinante. En effet, sur Github (\url{https://github.com/docker/docker}) les commits se succèdent à vitesse grand V. Entre chaque version de Docker qui sortent environ tous les mois, il est courant d'avoir plus de 3000 commits qui ont été \emph{pushés}. Tout ceci pour dire qu'à la lecture de ce document, il est quasiment sûr que certaines pistes explorées soient définitivement obsolètes ou au contraire deviennent la voie à suivre du à une mise à jour quelconque.


\section{Objectifs}

De manière plus précise, ce projet vise à maitriser les parties suivantes:

\begin{enumerate}
  \item Construction d'un système Linux capable de faire tourner Docker et son \emph{daemon} en utilisant Buildroot. Pour générer le dit système, on dispose d'un \emph{repository} Gitlab hébergé à la Haute École de Fribourg

  \item Cross compilation de Docker et de son \emph{daemon}, capable de faire tourner des containers
  
  \item Comprendre, analyser et évaluer l'aspect sécurité de Docker dans le cadre d'une utilisation avec une carte embarquée
\end{enumerate}

Les deux premiers points ont été traités dans un précédent rapport appelé "État de l’art à la mi-projet de semestre Docker and embedded systems - Ou comment ne pas cross compiler Docker sur ARM".

Ce document se concentre, dès lors, sur le dernier point ainsi que sur le déroulement global du projet.


\chapter{Objectif 3 - TODO}

TODO TODO TODO TODO TODO TODO TODO TODO TODO 

\section{TODO}


\begin{appendices}
  \chapter{TODO}

\end{appendices}
\end{document}
